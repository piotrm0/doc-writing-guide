\paragraph{Comprehension and Style}
\begin{itemize}
\item{} When discussing a term for the first time, describe its function before going into further
  details. For example:

  \textit{"We first introduce a concept of \textit{quantity of interest}
    (QoI)~\cite{leino2018influence} as one of the building blocks for attributions. QoI is
    represented as a continuous mapping from the input space to a scalar value or vector, which
    identifies the question an explanation tool aims to answer."}

  In this case, the representation is noted before its purpose. Consider switching them around.

\item{} Each paragraph should feature one main point and supporting sentences. As a drafting
  exercise, summarize each paragraph using a single simple sentence. If you cannot, consider
  breaking it apart.

  \begin{itemize}
    \item{} The summary sentences should be mostly readable in that you can get a sense of the
      paper just by reading the summary sentences and nothing else. Related, please fill in the
      summary sentences as they help your proof-readers understand the purpose of each part of the
      paper and give proper feedback.
    \item{} Say what needs to be said and nothing more. What needs to be said is what you summarize
      in the summary sentences.
  \end{itemize}

\item{} \textbf{Comma use}: use a comma in places where you pause reading. If there is no pause,
  there should be no comma.

\item{} Drop any word which is not essential. Examples:
  \begin{itemize}
  \item{} ``We can observe that 2+2=4'' can become ``We observe that 2+2=4'' can become ``Note that
    2+2=4'' can sometimes become ``2+2=4''.
  \item{} You do not need to mention \textit{"In this paper we focus on ..."}. You can avoid
    \textit{"In this paper"} overall and try to avoid "we focus on" or similar as well. \textit{"In
      this paper we do X"} becomes \textit{"We do X"} or when possible just \textit{X}.
  \item{} ``In order to X'' becomes ``To X''.
  \end{itemize}
\end{itemize}

\paragraph{Writing for Peer Review}
\begin{itemize}
\item{} Do not make opinion statements especially when referring to other papers. Also try not to
  make points about who was first to whatever lest a reviewer disagrees.

\item{} Avoid adjectives especially when they indicate opinion. Example:

  \textit{While we believe that overfitting plays a major role in membership inference attacks, our
    investigation reports an interesting observation.}

\item{} Avoid over-emphasizing a single related work. You do not want your work to be rejected
  because someone believes X is not valid work and might if they feel like you are building
  upon it too centrally.
\end{itemize}

\paragraph{Figures}
\begin{itemize}
  \item{} Changing font sizes to fit more text is frowned upon. Avoid this if possible. It is ok
    for captions to be long. Shrinking table content text may be acceptable. This is tricky for
    figures. If you are using matplotlib, you can use the following approach to unify font sizes
    between matplotlib and latex. First, in matplotlib, specify figure size to be what its size in
    the paper will be and then \textbf{NOT} use the \verb|scale| option in \verb|includegraphics|.

    You can check how much space a figure should take up with \verb|\printlength{\linewidth}|, for
    example, here this returns linewidth=\printlength{\linewidth}. Make your matplotlib figures
    this size:
\begin{verbatim}
\includegraphics[width=...]{...}
\end{verbatim}

\begin{verbatim}
\usepackage{printlen}\uselengthunit{in}
\end{verbatim}

\end{itemize}

\paragraph{Formalisms}
\begin{itemize}
\item{} Distinguish definitions from equations. One option to do this is use a different symbol
  for definitions like $ \defeq $ as defined:

\begin{verbatim}
\newcommand{\stacklabel}[1]{%
  \stackrel{\smash{%
    \scriptscriptstyle \mathrm{#1}}%
  }%
}
\newcommand{\defeq}{\stacklabel{def}=}
\end{verbatim}

\item{} Try to use standard notation whenever possible. If standard notation is not available in
  your field, try to find related notation from nearby formalisms. Whatever you do, stay
  consistent.
\end{itemize}

\paragraph{Citations}
\begin{itemize}
\item{} Avoid referring to works as nouns and instead add a citation to a point being made. It
  is preferable to avoid the statement referring to the work or author and instead discuss what
  you wanted to mention about that work, adding a citation. For example:

  \textbf{Original:} \textit{Shokri et al. introduced the concept of Shadow model, which are
    essentially replicates of the target model. }

  \textbf{Better:} \textit{The shadow models attack constructs shadow models to replicate the
    target model (Shokri et al.). }

\item Use citation capabilities in latex. Do not manage the bibliography yourself. The
  \texttt{natbib} package features three citation commands. \verb|cite| as in
  ``\cite{leino2018influence}'', \verb|citet| as in ``\citet{leino2018influence}'', and
  \verb|citep| as in ``\cite{leino2018influence}''.

  \item{} Citations do not excuse plagiarism. If you are copying text, it needs to be quoted.
    Technical writing rarely requires quotation so use sparingly. Guidelines regarding quotation
    and plagiarism are available:
    \url{https://writing.wisc.edu/handbook/assignments/quotingsources/}. Note that you also need to
    quote yourself if copying text from another work. Again, this should be rare. Also note
    difficulties in this regard in the Anonymity section.

  \item{} Make sure the same works are not cited as multiple items in the bibliography. For works
    available in preprint (arXiv, tech-report, etc) and peer-reviewed venue simultaneously, include
    only the peer-reviewed version in the bibliography.


\end{itemize}

\paragraph{Anonymity}
\begin{itemize}
  \item{} Your own works need to be referred to in third party. Links that may de-anonymize you
    (software, artifacts, etc), need to be removed or if needed for peer-review, anonymized.
\end{itemize}

\paragraph{Experiments and Artifacts}
\begin{itemize}
  \item{} Include enough information to replicate an experiment or at least run an experiment
    demonstrating the same point. This means that not all details need to be presented but
    parameters being investigated do. Options that could have an impact on the conclusions made
    need to be included as well. It is often difficult to determine these so software artifacts
    with exact experimental setups are preferable.
  \item{} If you include benchmarking results, also include information necessary to replicate them
    such as the hardware involved.
\end{itemize}
